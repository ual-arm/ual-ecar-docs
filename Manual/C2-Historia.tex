%%%%%%%%%%%%%%%%%%%%%%%%%%%%%%%%%%%%%%%%%%%%%%
% Todos los proyectos realizados sobre el vehículo
%%%%%%%%%%%%%%%%%%%%%%%%%%%%%%%%%%%%%%%%%%%%%%
\chapter{Trabajos anteriores}
\section{Trabajos Fin de Grado}
\textbf{Titulo:} \textit{Modelado del sistema de almacenamiento de energía y del controlador del motor para un vehículo eléctrico Tesur} \cite{kurucz2014Tesur}.

\textbf{Autor:} Kurucz , Adrian Paul.

\textbf{Director:} Rodríguez-Díaz, Francisco De Asís / Blanco-Claraco, José Luis.

\textbf{Grado:} Ingeniería Electrónica Industrial. \textbf{Curso:} 2013-14. 

\textbf{Ficha:} \url{http://repositorio.ual.es/handle/10835/3399}
 
\textbf{Resumen:} La situación energética actual y futura está condicionada por las limitadas reservas de combustibles fósiles, a lo que se suma la creciente preocupación por el medio ambiente y la eficiencia energética. Esta situación afecta especialmente al sector transportes, ya que es el mayor consumidor de energía. Los vehículos eléctricos se presentan como una solución prometedora a los problemas con los que se enfrenta el sector del transporte. Pero estos a su vez están condicionados por la energía eléctrica disponible a bordo, es decir, por el estado de carga de las baterías (SOC). Este parámetro no es medible, por lo cual es necesario estimarlo basándose en las mediciones de otras señales disponibles en las baterías, tales como tensión, corriente y temperatura. Este trabajo presenta un modelo eléctrico de las baterías, capaz de estimar el SOC, la tensión de las baterías y otros parámetros de interés a partir de la intensidad de descarga y la temperatura del electrólito de las baterías. Además, se ha realizado un estudio y análisis de tipos de baterías, centrándose en las baterías de Pb-ácido, sus características, modos de fallos, etc. La energía disponible a bordo se debe utilizar de forma eficiente, por ello en este trabajo también se procede a la caracterización del controlador del motor de impulsión de un vehículo eléctrico en función de las consignas de aceleración. 

%%%%%%%%%%%%%%%%%%%%%%%%%%%%%%%%%%%%%%%%%%%%%%%%%%%%%%%%%%%%%%%%%%%%%%%%%%%%%%%%%%%%%%%%%
\vspace{6pt} \hrule \vspace{6pt}
%%%%%%%%%%%%%%%%%%%%%%%%%%%%%%%%%%%%%%%%%%%%%%%%%%%%%%%%%%%%%%%%%%%%%%%%%%%%%%%%%%%%%%%%%
 
 \textbf{Titulo:} \textit{Modelado y control de la dirección de un vehículo eléctrico} \cite{ramos2014direccion}.

\textbf{Autor:} Ramos Teodoro, Jerónimo.

\textbf{Director:} Blanco-Claraco, José Luis / Rodríguez-Díaz, Francisco De Asís.

\textbf{Grado:} Ingeniería Mecánica. \textbf{Curso:} 2013-14. 

\textbf{Ficha:} \url{http://repositorio.ual.es/handle/10835/3408}
 
\textbf{Resumen:} En la actualidad, el uso creciente del automóvil unido al desarrollo de Sistemas Inteligentes de Transporte, fomenta la investigación en el ámbito de los vehículos autónomos. Este trabajo tiene por objetivo el diseño de un sistema para el control de bajo nivel de la dirección de un vehículo eléctrico, siguiendo un esquema en cascada para el control lateral del mismo. Para ello, se instalan y calibran los elementos de medición y actuación que permiten el control en lazo cerrado del ángulo de Ackermann, dado por la posición de las ruedas directrices del vehículo, y se elabora un modelo de diagrama de bloques con la herramienta Simulink de MATLAB que se ajusta al comportamiento del sistema. Una vez validado el modelo, se diseña e implementa el sistema adecuado para el control Proporcional Integral Derivativo de la posición angular y se somete a pruebas de funcionamiento en un entorno de simulación. La principal aportación, además de poder integrar el modelo construido en Simulink en un modelo global del vehículo, consiste en que proporciona la base para la aplicación de algoritmos de navegación autónoma sobre el vehículo.

%%%%%%%%%%%%%%%%%%%%%%%%%%%%%%%%%%%%%%%%%%%%%%%%%%%%%%%%%%%%%%%%%%%%%%%%%%%%%%%%%%%%%%%%%
\vspace{6pt} \hrule \vspace{6pt}
%%%%%%%%%%%%%%%%%%%%%%%%%%%%%%%%%%%%%%%%%%%%%%%%%%%%%%%%%%%%%%%%%%%%%%%%%%%%%%%%%%%%%%%%%

\textbf{Titulo:} \textit{Diseño y construcción de un banco de ensayo para la caracterización del motor de un vehículo eléctrico} \cite{acuna2014motor}.

\textbf{Autor:} Acuña-Prieto, Francisco.

\textbf{Director:} Giménez-Fernández, Antonio / Torres-Moreno, José Luis.

\textbf{Grado:} Ingeniería Mecánica. \textbf{Curso:} 2014-15.

\textbf{Resumen:} El sector del transporte ha aumentado su demanda en los últimos años y se ha convertido en el sector con mayor consumo de energía. Esto ha provocado que, actualmente, la industria se centre en desarrollar vehículos más fiables y respetuosos con el medio ambiente. Este proyecto tiene por objetivo el diseño y construcción de un banco de ensayo capaz de realizar las mediciones necesarias para la caracterización de un motor de un vehículo eléctrico. Para ello, se realiza un diseño CAD de los elementos que requiere el banco para su funcionamiento y se fabrica la estructura. Posteriormente, se replica el sistema eléctrico del vehículo para adaptarlo al banco, y se monta la instrumentación. Finalmente, se realizan algunos ajustes a la configuración del controlador del motor, se diseña un sistema de adquisición de datos y se realizan los ensayos. La principal aportación del proyecto, además de crear una herramienta que permita obtener los parámetros para realizar el control de un motor eléctrico, consiste en ofrecer la posibilidad de, realizando algunas modificaciones, simular diferentes situaciones que se encontraría el vehículo eléctrico en la realidad.
 
%%%%%%%%%%%%%%%%%%%%%%%%%%%%%%%%%%%%%%%%%%%%%%%%%%%%%%%%%%%%%%%%%%%%%%%%%%%%%%%%%%%%%%%%%
\vspace{6pt} \hrule \vspace{6pt}
%%%%%%%%%%%%%%%%%%%%%%%%%%%%%%%%%%%%%%%%%%%%%%%%%%%%%%%%%%%%%%%%%%%%%%%%%%%%%%%%%%%%%%%%%
 
\textbf{Titulo:} \textit{Análisis e instalación de un sistema fotovoltaico en el vehículo eléctrico UAL-eCARM} \cite{guerrero2015solar}.

\textbf{Autor:} Guerrero-Cabezas, Francisco Javier.

\textbf{Director:} Rodríguez-Díaz, Francisco De Asís.

\textbf{Grado:} Ingeniería Mecánica. \textbf{Curso:} 2015-16. 

%\textbf{Ficha:} \url{http://repositorio.ual.es/handle/}
 
\textbf{Resumen:} En la actualidad, el desarrollo de la tecnología, la reducción de costes en instalaciones y la concienciación social acerca de los problemas medioambientales han impulsado el interés por las fuentes de energía renovables cuya principal ventaja competitiva es su concepción como fuente inagotable. El objetivo de este trabajo es por lo tanto, el estudio de las características e instalación de un sistema fotovoltaico aislado como fuente de energía alternativa y/o complementaria a la red eléctrica en un vehículo eléctrico y el análisis de su eficiencia energética. Siendo así, se ha calculado y diseñado un equipo de medida para analizar un Sistema Fotovoltaico Aislado (SFA) en régimen de corriente continua, con objeto de estudiar las Curvas Tensión-Intensidad características. Posteriormente a los resultados de dicho estudio y con el conocimiento de las distintas condiciones que supeditan su funcionamiento, se ha llevado a cabo la instalación del SFA en el vehículo eléctrico y se ha dispuesto la instrumentación electrónica necesaria para llevar a cabo el estudio de los procesos de carga y descarga de los acumuladores presentes. La principal aportación ha consistido en facilitar el acceso al conocimiento de los factores determinantes en la eficiencia de un Sistema Fotovoltaico mediante el equipo diseñado y acotar un camino hacia la búsqueda de soluciones para dotar de mayor trascendencia a las ventajas propias de la energía solar fotovoltaica.
 
%%%%%%%%%%%%%%%%%%%%%%%%%%%%%%%%%%%%%%%%%%%%%%%%%%%%%%%%%%%%%%%%%%%%%%%%%%%%%%%%%%%%%%%%%
\vspace{6pt} \hrule \vspace{6pt}
%%%%%%%%%%%%%%%%%%%%%%%%%%%%%%%%%%%%%%%%%%%%%%%%%%%%%%%%%%%%%%%%%%%%%%%%%%%%%%%%%%%%%%%%% 

\textbf{Titulo:} \textit{Estudio de la dirección de un vehículo eléctrico mediante herramientas CAD-CAE} \cite{perez2017CAD}.

\textbf{Autor:} Pérez-Candela, Alberto José.

\textbf{Director:} Torres-Moreno, José Luis.

\textbf{Grado:} Ingeniería Mecánica. \textbf{Curso:} 2016-17. 

%\textbf{Ficha:} \url{http://repositorio.ual.es/handle/}
 
\textbf{Resumen:} El sector automovilístico es conocido como un sector con un gran potencial y rentabilidad. Esto unido a los crecientes avances en los Sistemas Inteligentes de Transporte, invita al desarrollo y el estudio de este nicho de oportunidades que se nos abre. Este proyecto tiene por objetivo la implementación de un sistema de control de la dirección y de la amortiguación del vehículo eléctrico de la UAL eCARM, conectando un modelo tridimensional a este, consiguiendo de este modo los parámetros de todo el sistema de dirección y reduciendo así el número de sensores que debería tener el vehículo. Para ello, primero se mide y modela en 3D con el programa SolidWorks respetando su posición original y aplicando las condiciones de contorno convenientes. En segundo lugar se combina dicho modelo con un sistema SCADA que permite conseguir información a partir del modelo además de realizar su control. Finalmente, la creación de un prototipo a base de actuadores y sensores que se comunican con el software y permite su control y la reducción del error mediante lo que se conoce como “Hardware in Loop” (HIL). La principal aportación es la integración el modelo 3D construido en SolidWorks con LabView para su cosimulación, lo que permite la reducción de sensores para el conocimiento de la posición y el control del sistema de dirección. Además de esta aportación, la otra aportación es la creación de un prototipo con las mismas dimensiones que el eCARM y que permita la simulación del movimiento de la dirección y la amortiguación.

%%%%%%%%%%%%%%%%%%%%%%%%%%%%%%%%%%%%%%%%%%%%%%%%%%%%%%%%%%%%%%%%%%%%%%%%%%%%%%%%%%%%%%%%%
\vspace{6pt} \hrule \vspace{6pt}
%%%%%%%%%%%%%%%%%%%%%%%%%%%%%%%%%%%%%%%%%%%%%%%%%%%%%%%%%%%%%%%%%%%%%%%%%%%%%%%%%%%%%%%%%

\textbf{Titulo:} \textit{Caracterización y control de sistema drive-by-wire en vehículo eléctrico}\cite{manas2017UALeCARM}.

\textbf{Autor:} Mañas-Álvarez, Francisco José.

\textbf{Director:} Blanco-Claraco, José Luis / Guzmán-Sánchez, José Luis.

\textbf{Grado:} Ingeniería Electrónica Industrial. \textbf{Curso:} 2016-17. 

\textbf{Ficha:} \url{http://repositorio.ual.es/handle/10835/6436}
 
\textbf{Resumen:} La elevada demanda de vehículos, el progreso tecnológico asociado a la automatización de nuestro entorno y la preocupación por la conservación del medio ambiente está guiando la investigación de vehículos hacia mejores resultados con vehículos eléctricos y autónomos. Pero el control a tan alto nivel no es posible si en los niveles inferiores, los sistemas de control no son lo suficientemente eficaces. Este trabajo presenta como objetivo la realización de un control eficaz a bajo nivel sobre los sistemas Steer-by-wire y Throttle-by-wire. Para ello, en primer lugar se debe realizar el modelado de los sistemas actuales del vehículo. Una vez se comprueba la validez de los modelos obtenidos, se trabaja en el diseño de los sistemas de control mediante herramientas de simulación, en este caso se emplea principalmente Simulink. Los principales problemas tratados en el trabajo son la identificación de modelos de alto orden, en el caso del sistema Throttle-by-wire, y el control en cascada con un sistema de retardo dominante en el lazo interno para el caso del sistema Steer-by-wire. Posteriormente, con el sistema de control ya diseñado y comprobada su eficiencia en simulación, se implementa en el sistema real incorporándolo al sistema ya existente creado en ROS. La principal aportación de este trabajo, además de la actualización de los modelos instalados en el vehículo eCARM, es la implementación del sistema de control completo en el nodo steer\_controller del sistema ROS.

%%%%%%%%%%%%%%%%%%%%%%%%%%%%%%%%%%%%%%%%%%%%%%%%%%%%%%%%%%%%%%%%%%%%%%%%%%%%%%%%%%%%%%%%%
\vspace{6pt} \hrule \vspace{6pt}
%%%%%%%%%%%%%%%%%%%%%%%%%%%%%%%%%%%%%%%%%%%%%%%%%%%%%%%%%%%%%%%%%%%%%%%%%%%%%%%%%%%%%%%%%
 
\textbf{Titulo:} \textit{Integración y caracterización del sistema fotovoltaico en el vehículo UAL-eCARM} \cite{poyatos2019UALeCARM}.

\textbf{Autor:} Poyatos Bakker, Aaron Raúl.

\textbf{Director:} Blanco-Claraco, José Luis / Torres-Moreno, José Luis.

\textbf{Grado:} Ingeniería Mecánica. \textbf{Curso:} 2018-19. 

%\textbf{Ficha:} \url{http://repositorio.ual.es/handle/}
 
\textbf{Resumen:} Añadir resumen.
 
%%%%%%%%%%%%%%%%%%%%%%%%%%%%%%%%%%%%%%%%%%%%%%%%%%%%%%%%%%%%%%%%%%%%%%%%%%%%%%%%%%%%%%%%%
\vspace{6pt} \hrule \vspace{6pt}
%%%%%%%%%%%%%%%%%%%%%%%%%%%%%%%%%%%%%%%%%%%%%%%%%%%%%%%%%%%%%%%%%%%%%%%%%%%%%%%%%%%%%%%%%
 
 \textbf{Titulo:} \textit{Título}. %\cite{}.

\textbf{Autor:} Autor.

\textbf{Director:} Director.

\textbf{Grado:} Ingeniería . \textbf{Curso:} 20xx-xx. 

%\textbf{Ficha:} \url{http://repositorio.ual.es/handle/}
 
\textbf{Resumen:} Resumen.
 
\section{Trabajos Fin de Máster}
\textbf{Titulo:} \textit{Estudio del comportamiento dinámico de un vehículo eléctrico mediante SimMechanics} \cite{torres2011UALeCARM}.

\textbf{Autor:} Torres-Moreno, José Luis.

\textbf{Director:} Giménez-Fernández, Antonio.

\textbf{Máster:} Informática Industrial . \textbf{Curso:} 2010-11. 

\textbf{Ficha:} \url{http://hdl.handle.net/10835/1187}
 
\textbf{Resumen:} The introduction of electric propulsion systems in automobiles is helping to reduce fossil fuel consumption and optimize the e ciency of new vehicles. This paper analyzes and simulates the dynamic behavior of an electric vehicle. Dynamic equations are formulated and a three-dimensional prototype is built, which allows the collection of data on mass and inertia of its components. All these variables are implemented in a model that is analyzed by the tool for the analysis of Multibody systems SimMechanics. The main contribution of this paper is the proposal, once validated the model, of a change in the distribution of mass of the vehicle which improves the dynamic performance of it. And thanks to the integration of this model in MATLAB / Simulink, future additions such as navigation systems, autonomous control, brake assist and stability control, among others are possible.
 
%%%%%%%%%%%%%%%%%%%%%%%%%%%%%%%%%%%%%%%%%%%%%%%%%%%%%%%%%%%%%%%%%%%%%%%%%%%%%%%%%%%%%%%%%
\vspace{6pt} \hrule \vspace{6pt}
%%%%%%%%%%%%%%%%%%%%%%%%%%%%%%%%%%%%%%%%%%%%%%%%%%%%%%%%%%%%%%%%%%%%%%%%%%%%%%%%%%%%%%%%%

\textbf{Titulo:} \textit{Modelado y control multivariable del vehículo urbano eléctrico UAL-eCARM} \cite{manas2019UALeCARM}.

\textbf{Autor:} Mañas-Álvarez, Francisco José.

\textbf{Director:} Blanco-Claraco, José Luis / Rodríguez-Díaz, Francisco De Asís.

\textbf{Grado:} Ingeniería Industrial. \textbf{Curso:} 2018-19. 

%\textbf{Ficha:} \url{http://repositorio.ual.es/handle/}
 
\textbf{Resumen:} Los sistemas de gran complejidad, como los vehículos, necesitan que todos sus elementos funcionen bien tanto de forma individual como en conjunto. Si la sinergia entre los distintos subsistemas es elevada, mejor es el desempeño de los resultados obtenidos. Este trabajo presenta como objetivo la mejora de la plataforma UAL-eCARM, un vehículo eléctrico urbano. El trabajo comprende el modelado y control de los sistemas de dirección y tracción del vehículo. Para ello se han mejorado los modelos y arquitecturas de control del mecanismo Steer-by-Wire. El sistema Throttle-by-Wire, un sistema no lineal, ha sido modelado como un sistema lineal entorno a cuatro puntos de operación. La estrategia de control implementada es un controlador PI adaptativo según el ajuste de Ziegler-Nichols. El sistema Brake-by-Wire, el más reciente instalado, se ha modelado y controlado mediante un controlador PI ajustado manualmente. La implementación está disponible en Github. Las principales aportaciones de este trabajo son la ampliación y mejora de los modelos del vehículo UAL-eCARM y la migración de la arquitectura de control desde los nodos de ROS a
microcontroladores.
 
%%%%%%%%%%%%%%%%%%%%%%%%%%%%%%%%%%%%%%%%%%%%%%%%%%%%%%%%%%%%%%%%%%%%%%%%%%%%%%%%%%%%%%%%%
\vspace{6pt} \hrule \vspace{6pt}
%%%%%%%%%%%%%%%%%%%%%%%%%%%%%%%%%%%%%%%%%%%%%%%%%%%%%%%%%%%%%%%%%%%%%%%%%%%%%%%%%%%%%%%%%

\textbf{Titulo:} \textit{Título}. %\cite{}.

\textbf{Autor:} Autor.

\textbf{Director:} Director.

\textbf{Máster:} --- . \textbf{Curso:} 20xx-xx. 

%\textbf{Ficha:} \url{http://repositorio.ual.es/handle/}
 
\textbf{Resumen:} Resumen.
 
%%%%%%%%%%%%%%%%%%%%%%%%%%%%%%%%%%%%%%%%%%%%%%%%%%%%%%%%%%%%%%%%%%%%%%%%%%%%%%%%%%%%%%%%%
\vspace{6pt} \hrule \vspace{6pt}
%%%%%%%%%%%%%%%%%%%%%%%%%%%%%%%%%%%%%%%%%%%%%%%%%%%%%%%%%%%%%%%%%%%%%%%%%%%%%%%%%%%%%%%%%

\section{Doctorados}
\textbf{Titulo:} \textit{Análisis multidominio de vehículos eléctricos} \cite{torres2014UALeCARM}.

\textbf{Autor:} Torres-Moreno, José Luis.

\textbf{Director:} Giménez-Fernández, Antonio / Blanco-Claraco, José Luis.

\textbf{Doctorado:} En Informática . \textbf{Año:} 2014. 

%\textbf{Ficha:} \url{http://repositorio.ual.es/handle/}
 
\textbf{Resumen:} El automóvil es uno de los productos tecnológicos consumidos a gran escala que abarca un mayor número de sistemas mecánicos, eléctricos y electrónicos. Todos estos sistemas son complejos y afectan de forma individual o interactuando entre ellos al funcionamiento del vehículo en su conjunto. Además, el número de sistemas de control que incorporan los coches modernos cada vez es mayor. Para poder diseñar un controlador es preciso conocer en primer lugar cómo se comportan los sistemas que determinan el funcionamiento del vehículo. Desde esta tesis se proponen una serie de metodologías para analizar dichos sistemas. Estas metodologías son muy diferentes dependiendo del sistema del automóvil que se estudie. Por tanto se analizará cada una de ellas por separado.

Dada la inminente aparición masiva de vehículos con motores diferentes de los de combustión interna (en combinación con éstos o sustituyéndolos), se hace necesario un estudio de este tipo de vehículos. Entre las opciones más destacadas se encuentran los denominados vehículos híbridos eléctricos, cuyas particularidades implican atender a aspectos que no estaban presentes en los vehículos convencionales, como puede ser el control de un sistema de propulsión totalmente renovado o el comportamiento dinámico derivado de una distribución de masas diferente a la de los vehículos de gasolina. Algunas de las metodologías propuestas en esta tesis están orientadas a analizar estas características, incluyendo el diseño de una estrategia de control de gestión energética para la utilización conjunta de las diferentes fuentes de potencia incorporadas en vehículos híbridos eléctricos.

En las fases tempranas del diseño de un vehículo se prefiere, en la medida de lo posible, trabajar con modelos en lugar de con prototipos físicos. Por este motivo, en esta tesis se exploran varias alternativas que permitan la realización de experimentos mediante simulaciones. Entre los modelos considerados aparecen, por un lado, los modelos multicuerpo, que pueden alcanzar un alto nivel de realismo, y los modelos simplificados, ideales para diseñar controladores y observadores de estado. Esto no exime de la necesidad de realizar experimentos reales con un prototipo. Este, deberá de incorporar una serie de sensores y actuadores que permitan medir los parámetros que mejor definen el su comportamiento. Además es necesario contar con una arquitectura de control que se encargue de gestionar las comunicaciones que tienen lugar en las diferentes capas. En esta tesis también se presenta la metodología llevada a cabo para la construcción de una arquitectura que proporciona una capa de abstracción en el vehículo que facilita la implementación de algoritmos de control.
 
%%%%%%%%%%%%%%%%%%%%%%%%%%%%%%%%%%%%%%%%%%%%%%%%%%%%%%%%%%%%%%%%%%%%%%%%%%%%%%%%%%%%%%%%%
\vspace{6pt} \hrule \vspace{6pt}
%%%%%%%%%%%%%%%%%%%%%%%%%%%%%%%%%%%%%%%%%%%%%%%%%%%%%%%%%%%%%%%%%%%%%%%%%%%%%%%%%%%%%%%%%
\section{Artículos}
\textbf{Titulo:} \textit{Benchmarking Particle Filter Algorithms for Efficient Velodyne-Based Vehicle Localization} \cite{blanco2019benchmarking}.

\textbf{Autores:} Blanco-Claraco, J. L., Mañas-Alvarez, F., Torres-Moreno, J. L., Rodriguez, F., y Gimenez-Fernandez, A.

\textbf{Revista:} Sensors 2019, 19(14), 3155; \url{https://doi.org/10.3390/s19143155}

\textbf{Resumen:} Keeping a vehicle well-localized within a prebuilt-map is at the core of any autonomous vehicle navigation system. In this work, we show that both standard SIR sampling and rejection-based optimal sampling are suitable for efficient (10 to 20 ms) real-time pose tracking without feature detection that is using raw point clouds from a 3D LiDAR. Motivated by the large amount of information captured by these sensors, we perform a systematic statistical analysis of how many points are actually required to reach an optimal ratio between efficiency and positioning accuracy. Furthermore, initialization from adverse conditions, e.g., poor GPS signal in urban canyons, we also identify the optimal particle filter settings required to ensure convergence. Our findings include that a decimation factor between 100 and 200 on incoming point clouds provides a large savings in computational cost with a negligible loss in localization accuracy for a VLP-16 scanner. Furthermore, an initial density of aprox. 2 particles/m 2 is required to achieve 100\% convergence success for large-scale (aprox. 100,000 m 2 ), outdoor global localization without any additional hint from GPS or magnetic field sensors. All implementations have been released as open-source software.

%%%%%%%%%%%%%%%%%%%%%%%%%%%%%%%%%%%%%%%%%%%%%%%%%%%%%%%%%%%%%%%%%%%%%%%%%%%%%%%%%%%%%%%%%
\vspace{6pt} \hrule \vspace{6pt}
%%%%%%%%%%%%%%%%%%%%%%%%%%%%%%%%%%%%%%%%%%%%%%%%%%%%%%%%%%%%%%%%%%%%%%%%%%%%%%%%%%%%%%%%%

\textbf{Titulo:} \textit{Modelado y control multivariable del vehículo urbano eléctrico UAL-eCARM}\cite{manas2020byWire}.

\textbf{Autores:} Mañas-Álvarez, F.J., JBlanco-Claraco, J. L., Torres-Moreno, J. L. y Gimenez-Fernandez, A.

\textbf{Revista:} Revista Iberoamericana de Automática e Informática Industrial 2020, ??(??), ????; \url{https://doi.org/10.4995/riai.2019.12679}

\textbf{Resumen:}Este trabajo presenta el modelado y control completo del sistema Drive-by-Wire de un vehículo urbano eléctrico. Dicho sistema comprende el mecanismo de dirección, la aceleración y el freno del vehículo. El modelado se ha realizado empleando funciones de transferencia de bajo orden haciendo uso de modelos de “caja negra”. En lo referente al control, todos los controladores desarrollados son del tipo PID en sus distintas configuraciones. Los actuadores de corriente continua acoplados a la dirección y freno se controlan mediante un sistema de control en cascada mientras que la aceleración está controlada por un sistema de planificación de ganancias. El código del proyecto se encuentra disponible en la plataforma Github. Los resultados obtenidos demuestran la validez de los modelos obtenidos, así como la eficacia de los controladores desarrollados.

%%%%%%%%%%%%%%%%%%%%%%%%%%%%%%%%%%%%%%%%%%%%%%%%%%%%%%%%%%%%%%%%%%%%%%%%%%%%%%%%%%%%%%%%%
\vspace{6pt} \hrule \vspace{6pt}
%%%%%%%%%%%%%%%%%%%%%%%%%%%%%%%%%%%%%%%%%%%%%%%%%%%%%%%%%%%%%%%%%%%%%%%%%%%%%%%%%%%%%%%%%

\section{Congresos}
\textbf{Titulo:} \textit{Caracterización y control del sistema Steer-By-Wire en un vehículo eléctrico}\cite{manas2019CIBIM}.

\textbf{Autores:} Mañas-Álvarez, Francisco José, López-Velasquez, Julián, Blanco-Claraco, José Luis, Torres-Moreno, José Luis, Giménez-Fernández, Antonio y Acosta-Amaya, Gustavo Alonso.

\textbf{Congreso:} 14 Congreso Iberoamericano de Ingeniería Mecánica.

\textbf{Localización:} Cartagena, Colombia. \textbf{Año:} 2019. 

\textbf{Resumen:} En este trabajo se afronta la realización de un control eficaz a bajo nivel sobre el sistema Steer-by-wire del prototipo de vehículo eléctrico autónomo desarrollado en la Universidad de Almería. En primer lugar, se obtiene la dinámica del motor DC encargado de actuar sobre la cremallera de dirección. Obteniendo un modelo de primer orden con retardo dominante para el comportamiento de la velocidad de giro del motor, se plantea un sistema de control en cascada para el control de la posición. Los posibles errores por saturación de la señal de control se evitan gracias a la implementación de la estrategia anti-windup a ambos controladores. Para facilitar la labor de validación de modelos y ensayo del sistema de control a instalar, se ha trabajado en simulación con la herramienta Simulink. El vehículo experimental está dotado con un PC industrial, y las aplicaciones corren bajo Ubuntu y ROS. La implementación se ha realizado mediante un nodo ROS que soporta el algoritmo de control y se comunica con la arquitectura hardware. En vista de los resultados obtenidos, la arquitectura de control ha demostrado ser robusta.

\afterpage{\blankpage}