%%%%%%%%%%%%%%%%%%%%%%%%%%%%%%%%%%%%%%%%%%%%%%
% Todos los proyectos realizados sobre el vehículo
%%%%%%%%%%%%%%%%%%%%%%%%%%%%%%%%%%%%%%%%%%%%%%
\chapter{Historia/Revisión Bibliográfica}
Indicar trabajos, autores, años, y resumen. Citar debidamente en la bibliografía.
\section{Trabajos Fin de Grado}

\section{Trabajos Fin de Máster}

\section{Doctorados}

\section{Artículos}
\textbf{Titulo:} \textit{Benchmarking Particle Filter Algorithms for Efficient Velodyne-Based Vehicle Localization} \cite{blanco2019benchmarking}.

\textbf{Autores:} Blanco-Claraco, J. L., Mañas-Alvarez, F., Torres-Moreno, J. L., Rodriguez, F., y Gimenez-Fernandez, A.

\textbf{Revista:} Sensors 2019, 19(14), 3155; \url{https://doi.org/10.3390/s19143155}

\textbf{Resumen:} Keeping a vehicle well-localized within a prebuilt-map is at the core of any autonomous vehicle navigation system. In this work, we show that both standard SIR sampling and rejection-based optimal sampling are suitable for efficient (10 to 20 ms) real-time pose tracking without feature detection that is using raw point clouds from a 3D LiDAR. Motivated by the large amount of information captured by these sensors, we perform a systematic statistical analysis of how many points are actually required to reach an optimal ratio between efficiency and positioning accuracy. Furthermore, initialization from adverse conditions, e.g., poor GPS signal in urban canyons, we also identify the optimal particle filter settings required to ensure convergence. Our findings include that a decimation factor between 100 and 200 on incoming point clouds provides a large savings in computational cost with a negligible loss in localization accuracy for a VLP-16 scanner. Furthermore, an initial density of aprox. 2 particles/m 2 is required to achieve 100\% convergence success for large-scale (aprox. 100,000 m 2 ), outdoor global localization without any additional hint from GPS or magnetic field sensors. All implementations have been released as open-source software.

\textbf{Titulo:} \textit{Modelado y control multivariable del vehículo urbano eléctrico UAL-eCARM}\cite{manas2020byWire}.

\textbf{Autores:} Mañas-Álvarez, F.J., JBlanco-Claraco, J. L., Torres-Moreno, J. L. y Gimenez-Fernandez, A.

\textbf{Revista:} Revista Iberoamericana de Automática e Informática Industrial 2020, ??(??), ????; \url{https://doi.org/10.4995/riai.2019.12679}

\textbf{Resumen:}Este trabajo presenta el modelado y control completo del sistema Drive-by-Wire de un vehículo urbano eléctrico. Dicho sistema comprende el mecanismo de dirección, la aceleración y el freno del vehículo. El modelado se ha realizado empleando funciones de transferencia de bajo orden haciendo uso de modelos de “caja negra”. En lo referente al control, todos los controladores desarrollados son del tipo PID en sus distintas configuraciones. Los actuadores de corriente continua acoplados a la dirección y freno se controlan mediante un sistema de control en cascada mientras que la aceleración está controlada por un sistema de planificación de ganancias. El código del proyecto se encuentra disponible en la plataforma Github. Los resultados obtenidos demuestran la validez de los modelos obtenidos, así como la eficacia de los controladores desarrollados.

\section{Congresos}