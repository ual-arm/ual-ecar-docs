%%%%%%%%%%%%%%%%%%%%%%%%%%%%%%%%%%%%%%%%%%%%%%
% Sistema de dirección
%%%%%%%%%%%%%%%%%%%%%%%%%%%%%%%%%%%%%%%%%%%%%%
\chapter{Sistema de dirección}
Establecer todo el contexto físico del mecanismo de dirección (ecuaciones) sin considerar aproximaciones. ACKERMANN. Las aproximaciones se pueden indicar a posteriori. Leyendo esta sección, se debe ser un experto en este mecanismo.

Realizar la identificación por todos los métodos posibles y comparar los resultados obtenidos: Caja negra, primeros principios; funciones de transferencia, espacios de estados; ... Pros y contras de cada una. Disponer de todos los modelos.

Estudiar los distintos sensores que dan información y aplicar Kalman para obtener la mejor salida o algo así. Estudiar desde el punto de vista energético también.
\afterpage{\blankpage}